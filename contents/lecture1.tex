\section{September 25, 2024}

\subsection{Logistics}
Fortnightly quiz; fortnightly problem sets (alternating). Quizzes are \textbf{very easy}. Midterm during Halloween week. Final exam. Algebra by Artin is the textbook. Term paper on Waffle word game and permutation groups.

\subsection{Permutations and Transpositions}
\begin{definition}
    A \textbf{permutation} of a set $A$ is some bijection to itself, i.e.
    \begin{align}
        \sigma: A \longleftrightarrow A
    \end{align}
\end{definition}
\begin{definition}
    A \textbf{transposition} is a permutation that changes exactly two elements.
\end{definition}
\begin{lemma}
    Two is the minimum number of elements changed
\end{lemma}
\begin{proof}
    Changing one but not another is a contradiction, as that cannot be injective and therefore not bijective.
\end{proof}
\begin{lemma}
    Suppose $\sigma_1, \sigma_2$ are permutations. Then, $\sigma_1\sigma_2$ is another, and is also bijective. So, given some desired permutation, it can be written as a sequence of transpositions.
\end{lemma}
\begin{proof}
    The base case (two elements) is trivial. We can easily transpose them as needed. Suppose we can write some set of $n-1$ elements as a sequence of transpositions. Then, we can write $n$ elements as a sequence of transpositions:
    \begin{enumerate}
        \item If the $n$-th element is correctly mapped, then we hold it and permute the first $n-1$ elements, which works by the inductive hypothesis.
        \item If not, then we can transpose the $n$-th element such that it \textit{is} in the correct place, and then permute the first $n-1$ elements. \qedhere
    \end{enumerate}
\end{proof}
Note that the order of permutations does matter (simple examples were written out in class and can be seen in the notes). A group is some set with a law of composition. Permutations are a group.